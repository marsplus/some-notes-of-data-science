

The focus of this chapter is to estimate the average treatment effect:
    \begin{equation}
        \E[Y | do( T:= 1)] - \E[Y | do( T:= 0)].
    \end{equation}

A key concept in causal inference is \emph{confounding}, formally:
    \begin{equation}
        P(Y=y | do(T:=t)) \ne P(Y=y | T=t).
    \end{equation}
Intuitively, the above inequality means that the conditional distribution of $Y$ given $T$ doesn't equal to its interventional counterpart. Otherwise, we can directly estimate the average treatment effect from observational data.

\subsection{Propensity Score Matching (PSM)}
    PSM estimates the effects of treatment from observational data.
    Let $T \in \SET{0, 1}$ be a binary treatment variable.
    The propensity score---the probability of being treated given the features $x$---is defined below
        \begin{equation}
            e(x) = \E[T | X = x].
        \end{equation}
    PSM proceeds as follows:
        \begin{enumerate}
            \item Construct a dataset $D = \SET{(x_i, t_i)}_{i=1}^{\ndata}$ from observational data, where $t_i \in \SET{0, 1}$ denotes whether sample $x_i$ got treatment.
            \item Fit a machine learning model $f$ to predict the propensity score for each $x_i$. This can be done by using cross validation. Specifically, we divide $D$ into multiple folds. We then train $f$ on a single fold and use it to predict the propensity scores for the other folds. 
            The final propensity scores are the average over the folds. 
            
            \item After step 2, we have two datasets with propensity scores: 1) the set of treated samples, i.e., $D_1 = \Set{(x_i, \hat{e}_i)}{t_i=1}$ and 2) the set of controlled samples, i.e., $D_0 = \Set{(x'_i, \hat{e}'_i)}{t_i=0}$.
            We then match a sample in $D_1$ with the sample in $D_0$ that has the closest propensity score.
            Formally, this creates a new dataset of matched pairs, i.e., $D_m = \SET{(x_u, x'_v)}$, where $x'_v \in D_0$ has the closest propensity score with $x_u \in D_1$.
            
            \item The average treatment effect is the difference between the outcomes of the treated samples and the matched controls.
        \end{enumerate}
        

\subsection{Synthetic Control Method (SMC}
    Suppose we would like to evaluate the effect of a policy on pollution levels. 
    The policy is applied to region A at time $t_1$. 
    The regions B, C, D, E and F are not affected by the policy. 
    SCM constructs a synthetic control group by combining data from multiple untreated units (regions B-F) to create a composite unit that closely resembles the treated unit in the pre-treatment period. 

    Some notations:
        \begin{itemize}
            % \item $y_{A}$ is the vector of pre-treatment pollution levels of region A.
            % \item $y_{i}$ is the vector of pre-treatment pollution level of region $i$.
            \item $y^\prime_{A}$ is the vector of post-treatment pollution level of region A.
            \item $y^\prime_{i}$ is the vector of post-treatment pollution level of region $i$.
            \item $x_A$ is the pre-treatment covariates of region A (may include pre-treatment outcomes).
            \item $x_i$ is the pre-treatment covariates of region $i$ (may include pre-treatment outcomes).
            \item $V$ is a matrix to quantify the interactions between the covariates and outcomes. 
            \item $w$ is a weight vector to combine the untreated units. 
        \end{itemize}
    The covariates of the synthetic control is $x_s = \sum_{i}^{}{w_i x_i}$.
    SMC solves the following problem:
        \begin{equation}
            \begin{aligned}
                 w^\ast \in \argmin_{w} (x_A - x_s)^\top V (x_a - x_s).
            \end{aligned}
        \end{equation}
    The treatment effect is estimated as:
        \begin{equation}
            y^\prime_A - \sum_{i}^{}{w^\ast_i y^\prime_i}.
        \end{equation}