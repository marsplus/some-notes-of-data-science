\subsection{Common Activation Function}
    \begin{itemize}
        \item Sigmoid: it is rarely used in practice anymore because: 1) it is easily saturated at the left and right ends, which makes gradients zeros; 2) the output is not zero-centered (it is $0.5$ centered), which is undesirable.
        \item Tanh: it is zero-centered, but the saturation issues still exist.
        \item ReLU: it is popular nowadays. However, it could make some neurons ``dead'', i.e., not activated in most of the training process.
        \item Some others including Leaky ReLU, Maxout, etc. See: \url{https://cs231n.github.io/neural-networks-1/#actfun}.
    \end{itemize}


\subsection{Optimization in Deep Learning}
    \subsubsection{Local Minimum/Maximum and Saddle Points}
    Let the loss function at a point $\bm{x}$ be $L(\bm{x})$.
    Suppose $\bm{x}$ is the a point where the gradients become zeros, i.e., $\nabla_{\bm{x}}L(\bm{x}) = \bm{0}$.
    We can determine whether $\bm{x}$ is a local minimum/maximum from the Hessian matrix $\bm{H}$:
        \begin{itemize}
            \item If all eigenvalues of $\bm{H}$ are positive, then $\bm{x}$ is a local minimum.
            \item If all eigenvalues of $\bm{H}$ are negative, then $\bm{x}$ is a local maximum.
            \item If the eigenvalues of $\bm{H}$ include both positives and negatives, then $\bm{x}$ is a saddle point.
        \end{itemize}
        
    \subsubsection{Gradient Descent (GD) in One-dimensional Case}
    Consider a one-dimensional function: $f(x): \R \rightarrow \R$.
    Do a first-order Taylor expansion of $f(x + \epsilon)$ at $x$ gives us:
        \begin{equation}
                f(x + \epsilon) \approx f(x) + \epsilon f'(x) + O(\epsilon^2).
        \end{equation}
    We want to choose $\epsilon$ such that the value of $f(x + \epsilon)$ is decreased compared with the value of $f(x)$.
    A natural choice is $\epsilon = -\eta f'(x)$.
    The right-hand side becomes $f(x) - \eta (f'(x))^2 + O(\epsilon^2)$: when $\epsilon$ is not too large such that $O(\epsilon^2)$ is not significant, we have:
        \begin{equation}
            f(x - \eta f'(x)) \approx f(x) - \eta (f'(x))^2 \le f(x).
        \end{equation}
    This justifies the choice of setting $\epsilon$ to the negative of the first-order derivative.
    Notice that the value of the learning rate $\eta$ is important.
    If $\eta$ is too large and the second-order term $O(\epsilon^2)$ becomes significant, moving toward the direction $-\eta f'(x)$ may not decrease the function value.
    
    \subsubsection{Newton's Method}\label{newton-method}
        Consider a function $f(\bm{x}): \R^\nfeat \rightarrow \R$.
        We do a second-order Taylor expansion of $f(\bm{x} + \bm{\epsilon})$ at $f(\bm{x})$ as follows:
            \begin{equation}
                f(\bm{x} + \bm{\epsilon}) \approx f(\bm{x}) + \bm{\epsilon}^{\top} \nabla f(\bm{x}) + \frac{1}{2} \bm{\epsilon}^\top \nabla^2 f(\bm{x}) \bm{\epsilon}.
            \end{equation}
        When at a minimum, we have $\nabla_{\bm{\epsilon}} f(\bm{x}+\bm{\epsilon}) = \bm{0}$; otherwise, there must be a direction $\bm{\epsilon}'$ where we can further minimize the function.
        Substituting into the Taylor expansion leads to 
            \begin{equation}
                \nabla_{\bm{\epsilon}}\left[ f(\bm{x}) + \bm{\epsilon}^{\top} \nabla f(\bm{x}) + \frac{1}{2} \bm{\epsilon}^\top \nabla^2 f(\bm{x}) \bm{\epsilon}\right] = \bm{0} \implies \bm{\epsilon} = -\bm{H}^{-1} \nabla f(\bm{x}).
            \end{equation}
        As a result, the update rule in Newton's method is as follows:
            \begin{equation}
                \bm{x} \leftarrow \bm{x} - \eta \bm{H}^{-1} \nabla f(\bm{x}),
            \end{equation}
        which shows how the Hessian matrix comes into play.
    
    
    
    \subsubsection{GD, SGD, and Mini-Batch SGD}
        The loss function is $L = \frac{1}{\ndata}\sum_{i=1}^{\ndata}{f(\bm{x}_i)}$.
        \begin{itemize}
            \item GD: compute the gradients over the entire dataset, i.e., $\nabla L = \sum_{i=1}^{\ndata}{\nabla f(\bm{x}_i)}$, which costs $O(\ndata)$ and might be too expensive when \ndata is large.
            \item SGD: randomly pick an index $i \in \SET{1, \ldots, \ndata}$ and compute $\nabla f(\bm{x}_i)$. This costs $O(1)$ but may oscillate a lot.
            \item Mini-batch SGD: instead of picking a single index, mini-batch SGD picks a subset $\mathcal{M} \subseteq \SET{1, \ldots, \ndata}$ uniformly at random and then computes $\sum_{i \in \mathcal{M}}^{}{\nabla f(\bm{i})}$.
            There are two benefits of mini-batch SGD: 1) it is more computationally efficient than both GD and SGD; 2) it reduces variance by averaging the gradients over a mini-batch.
        \end{itemize}
        
    \subsubsection{Common Optimizers in DL}
        \begin{itemize}
            \item Momentum: the key idea is to leverage historical gradients. Let $\bm{g}_t$ be the gradient at step $t$:
                \begin{equation}
                    \begin{aligned}
                        \bm{v}_t & = \beta  \bm{v}_{t-1} + \bm{g}_t \text{ (recursively taking past gradients into account)}\\
                        \bm{\theta}_t & = \bm{\theta}_{t-1} - \eta_t \bm{v}_t,
                    \end{aligned}
                \end{equation}
            where $\beta \in (0, 1)$. By tuning the value of $\beta$, we can take either long- or short-range gradients into account.
            \item Adagrad: recall from Section~\ref{newton-method} that the inverse of the Hessian matrix is used to precondition the gradients. 
            However, computing the Hessian and its inverse is prohibitively expensive. 
            Adagrad uses the following updating rule:
                \begin{equation}
                    \begin{aligned}
                        \bm{s}_t & = \bm{s}_{t-1} + \bm{g}_t^2  \text{ ($\bm{s}_t$ keeps track of the variance of the gradients)}\\
                        \bm{\theta}_t & = \bm{\theta}_{t-1} - \frac{\eta}{\sqrt{\bm{s}_t + \bm{\epsilon}}} \odot \bm{g}_t,
                    \end{aligned}
                \end{equation}
            where $\bm{g}_t^2$ is element-wise squares and $\bm{\epsilon} > \bm{0}$ ensures we do not divide by zeros.
            The intuition is that we can use the variance of the gradients as a cheap proxy for the scale of the Hessian. 
            \item RMSProp: the $\bm{s}_t$ in Adagrad may grow very large. A simple fix of RMSProp is as follows with $\gamma \in (0, 1)$:
                \begin{equation}
                    \begin{aligned}
                        \bm{s}_t & = \gamma \bm{s}_{t-1} + (1 - \gamma) \bm{g}_t^2  \\
                        \bm{\theta}_t & = \bm{\theta}_{t-1} - \frac{\eta}{\sqrt{\bm{s}_t + \bm{\epsilon}}} \odot \bm{g}_t.
                    \end{aligned}
                \end{equation}
            Another benefit of RMSProp is that the learning rate $\eta$ is decoupled from the scaling of $\bm{g}_t$.
            \item Adam: the key idea is to keep track of both the momentum and the second-order moments of the gradients.
            The updating rule of Adam is as follows:
                \begin{equation}
                    \begin{aligned}
                        \bm{v}_t & = \beta_1 \bm{v}_{t-1} + (1 - \beta_1) \bm{g}_{t} \text{ (keep track of momentum)} \\
                        \bm{s}_t & = \beta_2 \bm{s}_{t-1} + (1-\beta_2) \bm{g}_t^2 \text{ (keep track of second-order moments)} \\
                        \hat{\bm{v}}_t & = \frac{\bm{v}_t}{1 - \beta_1^t}, \hat{\bm{s}}_t = \frac{\bm{s}_t}{1 - \beta_2^t} \text{ (re-normalization)} \\
                        \bm{\theta}_t & = \bm{\theta}_{t-1}  - \eta \frac{\hat{\bm{v}}_t}{\sqrt{\hat{\bm{s}}_t} +\bm{\epsilon}}.
                    \end{aligned}
                \end{equation}
        \end{itemize}
        
    


\subsection{Variational Autoencoder (VAE)}
Let $\bm{x}$ represent the input data and $\bm{z}$ the hidden values. 
VAE can be interpreted from a Bayesian perspective as follows:
    \begin{enumerate}
        \item We sample a hidden vector $\bm{z}$ from its prior distribution, i.e., $\bm{z} \sim p_{\bm{\theta}}(\bm{z})$.
        \item We generate a output $\bm{x}$ by sampling from the posterior distribution conditioned on the hidden vector from the previous step:  $\bm{x} \sim p_{\bm{\theta}}(\bm{x} | \bm{z})$.
    \end{enumerate}
With this two-step framework, a standard way to learn $\bm{\theta}$ is maximum likelihood estimation:
    \begin{equation}
        \bm{\theta}^\ast \in \argmax_{\bm{\theta}} \underbrace{ \int_{}^{}{p_{  \bm{\theta}}(\bm{x} | \bm{z}) p_{\bm{\theta}}(\bm{z})} d \bm{z} }_{\text{the likelihood of $\bm{x}$, i.e., $p_{\bm{\theta}}(\bm{x})$}}.
    \end{equation}
However, the integral is hard to evaluate as enumerating all $\bm{z}$ is difficult.
Instead, the VAE paper does the following:
    \begin{itemize}
        \item The encoder: learn a distribution $q_{\bm{\phi}}(\bm{z} | \bm{x})$, which enables us to sample a hidden vector $\bm{z}$ given an input $\bm{x}$.
        This is an application of \emph{variational inference}, i.e., using parameterized $q_{\bm{\phi}}(\bm{z} | \bm{x})$ to approximate the true posterior $p(\bm{z} | \bm{x})$.
        \item The decoder: learn another distribution $p_{\bm{\theta}}(\bm{x} | \bm{z})$, which can be used to generate an output given the sampled hidden vector $\bm{z}$.
    \end{itemize}
The learning becomes maximizing the following:
    \begin{equation}\label{vae:obj}
        \log p_{\bm{\theta}}(\bm{x}) - D_\text{KL}(q_{\bm{\phi}}(\bm{z} | \bm{x}) || p_{\bm{\theta}}(\bm{z} | \bm{x})) = 
        \underbrace{\E_{\bm{z} \sim q_{\bm{\phi}}(\bm{z} | \bm{x})}\left[ p_{\bm{\theta}}(\bm{x} | \bm{z}) \right]}_{(\ast)} - \underbrace{D_\text{KL}(q_{\bm{\phi}}(\bm{z} | \bm{x}) || p_{\bm{\theta}}(\bm{z}))}_{(\diamond)}.
    \end{equation}
The left-hand side is called Evidence Lower Bound (ELBO), as it is a lower bound of the log-likelihood of $\bm{x}$, i.e.,
    \begin{equation}
        \log p_{\bm{\theta}}(\bm{x}) - D_\text{KL}(q_{\bm{\phi}}(\bm{z} | \bm{x})  ||  p_{\bm{\theta}}(\bm{x} | \bm{z})) \le \log p_{\bm{\theta}}(\bm{x}).
    \end{equation}

\paragraph{How to optimize the ELBO?} VAE is actually a model with both hidden variables (i.e., $\bm{z}$) and learnable parameters (i.e., those parameters $\bm{\theta}, \bm{\phi}$, etc.). 
A conventional approach to learn such a model is by using the famous EM algorithm. 
However, with the \emph{reparameterization trick} and the power of automatic differentiation tools, we can use gradient descent methods to learn a VAE.
A brief description of the optimization is as follows:
    \begin{itemize}
        \item To maximize $(\ast)$, we sample a set of hidden vectors $\bm{z}$ (the encoding part), which enables us to generate a set of vectors $\bm{x}^\prime$ through the decoder. We then minimize the $L_2$ distance between $\bm{x}$ and the sampled $\bm{x}^\prime$.
        \item The minimization of $(\diamond)$ may have closed-form solution; one example is given  in the VAE paper.
    \end{itemize}
Check out this fantastic blog for more details: \url{https://lilianweng.github.io/lil-log/2018/08/12/from-autoencoder-to-beta-vae.html}.
Another example implementation based on PyTorch is: \url{https://github.com/ethanluoyc/pytorch-vae/blob/master/vae.py}.

\subsection{RNN, GRU and LSTM}
This section is based on the excellent book: \url{https://d2l.ai/chapter_recurrent-neural-networks/bptt.html}.
% Let us first consider a one-dimensional RNN encoded as follows:
%     \begin{equation}
%         \begin{aligned}
%             h_t & = f(x_t, h_{t-1}, w_h) \\
%             o_t & = g(h_t, w_o).
%         \end{aligned}
%     \end{equation}
% The first equation models the ``hidden-to-hidden'' transition, where $x_t$ is the input data to the hidden layer and $w_h$ the parameter of the transition.
% Notice that the parameter of the ``input-to-hidden'' transition is included into $w_h$ for now.
% The second equation models the ``hidden-to-output'' transition.
% The loss function evaluated on a $T$-step sequence is as follows:
%     \begin{equation}
%         L(x_1, \ldots, x_T, y_1, \ldots, y_T, w_h, w_o) = \frac{1}{T}\sum_{i=1}^{T}{\ell(y_t, o_t)}.
%     \end{equation}
% We need to compute two derivatives:
%     \begin{equation}
%         \begin{aligned}
%             \frac{\partial L}{\partial w_o} & = \frac{1}{T}\sum_{i=1}^{T}{\frac{\partial \ell(y_t, o_t)}{\partial w_o}} = \frac{1}{T}\sum_{i=1}^{T}{\frac{\partial \ell(y_t, o_t)}{\partial o_t} \frac{\partial g(h_t, w_o)}{\partial w_o}} \\
%             \frac{\partial L}{\partial w_h} & = \frac{1}{T}\sum_{i=1}^{T}{\frac{\partial \ell(y_t, o_t)}{\partial w_h}} = \frac{1}{T}\sum_{i=1}^{T}{\frac{\partial \ell(y_t, o_t)}{\partial o_t} \frac{\partial g(h_t, w_o)}{\partial h_t}} \frac{\partial h_t}{\partial w_h}.
%         \end{aligned}
%     \end{equation}
% The first derivative is direct to compute. 
% For the second derivative, the computation of $\frac{\partial h_t}{\partial w_h}$ is tricky, since it involves a recursive computation:
%     \begin{equation}\label{rnn:simple-grad}
%         \frac{\partial h_t}{\partial w_h} = \frac{\partial f(x_t, h_{t-1}, w_h)}{\partial w_h} + \frac{\partial f(x_t, h_{t-1}, w_h)}{\partial h_{t-1}} \frac{\partial h_{t-1}}{\partial w_h},
%     \end{equation}
% where $h_{t-1}$ depends on $w_h$ through its previous hidden state $h_{t-2}$.
% We abstract the above recursive structure as three sequences $\SET{a_t}, \SET{b_t}, \SET{c_t}$, satisfying the following:
%     \begin{equation}\label{rnn:simple-series}
%         \begin{aligned}
%             a_t & = b_t + c_t \cdot a_{t-1} \\
%             a_0 & = 0,
%         \end{aligned}
%     \end{equation}
% where $a_t = \frac{\partial h_t}{\partial w_h}$, $b_t =  \frac{\partial f(x_t, h_{t-1}, w_h)}{\partial w_h}$ and $c_t = \frac{\partial f(x_t, h_{t-1}, w_h)}{\partial h_{t-1}}$.
% By setting $t=1$, the second term on the right-hand side of \eqref{rnn:simple-grad} should be zero, since there is no hidden state before $h_1$; this gives us the initial condition $a_0 = 0$.
% For any $t \ge 1$ we expand \eqref{rnn:simple-series}, which leads to:
%     \begin{equation}\label{rnn:key}
%         a_t = b_t + \sum_{i=1}^{t-1}{ \left( \prod_{j=i+1}^{t}{c_j} \right) b_i }.
%     \end{equation}
% The above formulate will help us a lot when we actually compute the gradients of RNN.

Consider a RNN with the following transitions:
    \begin{equation}
        \begin{aligned}
            \bm{h}_t & = \bm{W}_{hx} \bm{x}_t + \bm{W}_{hh} \bm{h}_{t-1} \\
            \bm{o}_t & = \bm{W}_{qh} \bm{h}_t.
        \end{aligned}
    \end{equation}
The loss funciton is $L=\frac{1}{T}\sum_{i=1}^{T}{\ell(y_t, \bm{o}_t)}$.
The tricky part is to compute $\frac{\partial L}{\partial \bm{h}_t}$.
When $t=T$, this is easy:
    \begin{equation}
        \frac{\partial L}{\partial \bm{h}_T} = \text{prod}\left( \frac{\partial L}{\partial \bm{o}_T}, \frac{\partial \bm{o}_T}{\partial \bm{h}_T}  \right) = \bm{W}_{qh}^\top \frac{\partial L}{\partial \bm{o}_T},
    \end{equation}
where $\frac{\partial L}{\partial \bm{o}_T}$ is direct to compute. 
When $t < T$, the loss $L$ depends on $\bm{h}_t$ and $\bm{h}_{t+1}$ (and recursively $\bm{h}_{t+2}$, etc.):
    \begin{equation}\label{rnn-grad}
        \begin{aligned}
            \frac{\partial L}{\partial \bm{h}_t} & = \text{prod}\left( \frac{\partial L}{\partial \bm{h}_{t+1}}, \frac{\partial \bm{h}_{t+1}}{\partial  \bm{h}_{t}} \right) + \text{prod}\left( \frac{\partial L}{\partial \bm{o}_t} , \frac{\partial \bm{o}_t}{\partial \bm{h}_t} \right) \\
            & = \bm{W}_{hh}^\top \frac{\partial L}{\partial \bm{h}_{t+1}} + \bm{W}_{qh}^\top \frac{\partial L}{\partial \bm{o}_t} \\
            & = \bm{W}_{hh}^\top\left( \bm{W}_{hh}^\top \frac{\partial L}{\partial \bm{h}_{t+2}} +  \bm{W}_{qh}^\top \frac{\partial L}{\partial \bm{o}_{t+1}} \right) + \bm{W}_{qh}^\top \frac{\partial L}{\partial \bm{o}_t} \\
            & = \bm{W}_{hh}^\top \bm{W}_{hh}^\top \left( \bm{W}_{hh}^\top \frac{\partial L}{\partial \bm{h}_{t+3}} +  \bm{W}_{qh}^\top \frac{\partial L}{\partial \bm{o}_{t+2}}\right) + \bm{W}_{hh}^\top \bm{W}_{qh}^\top \frac{\partial L}{\partial \bm{o}_{t+1}}  + \bm{W}_{qh}^\top \frac{\partial L}{\partial \bm{o}_t} \\
            & = \left( \bm{W}_{hh}^\top \right)^{T-t} \bm{W}_{qh}^\top \frac{\partial L}{\partial \bm{o}_T} + \left( \bm{W}_{hh}^\top \right)^{T-t-1}\bm{W}_{qh}^\top \frac{\partial L}{\partial \bm{o}_{T-1}} +, \ldots, + \left( \bm{W}_{hh}^\top \right)^{T-T} \bm{W}_{qh}^\top \frac{\partial L}{\partial \bm{o}_t} \\
            & = \sum_{i=t}^{T}{\left( \bm{W}_{hh}^\top \right)^{T-i} \bm{W}_{qh}^\top  } \left( \frac{\partial L}{\partial \bm{o}_{T+t-i} } \right).
        \end{aligned}
    \end{equation}
The so-called ``gradient explosion'' and ``gradient vanishing'' manifest themselves from the last step of \eqref{rnn-grad}.
They result from repeated multiplications of $\bm{W}_{hh}$:
    \begin{itemize}
        \item Gradient explosion: $\left(\bm{W}_{hh}^\top \right)^{T-i}$ will become 
        very large when some eigenvalues of $\bm{W}_{hh}$ have absolute values greater than one, which makes $\frac{\partial L}{\partial \bm{h}_t}$ very large.
        \item Gradient vanishing: $\left(\bm{W}_{hh}^\top \right)^{T-i}$ will become very small when some eigenvalues of $\bm{W}_{hh}$ have absolute values less than one, which makes $\frac{\partial L}{\partial \bm{h}_t}$ very small.
        \item How to mitigate gradient explosion/vanishing: 1) gradient clipping, i.e., in each iteration projecting the gradients back to a $L_2$ norm ball; 2) detach the gradients after a fixed number of time steps.
    \end{itemize}
    
\subsubsection{GRU}
    Gated Recurrent Units (GRU) are designed to 1) capture long- and short-term dependencies and 2) mitigate gradient explosion/vanishing.
    \begin{itemize}
        \item Reset gates: used to reset the hidden states in order to capture short-term dependencies by forgetting long-term dependencies.
        \item Update gates: used to maintain long-term dependencies by deciding whether to update the hidden states; in the extreme case, the hidden states can keep unchanged throughout the whole sequence.
    \end{itemize}
    
\subsubsection{LSTM}
    LSTM was designed to solve similar issues as that of GRU.
    There are three gates in a LSTM: input gates, output gates and forget gates. 
    There is also a cell to store information called memory cells.
    Check out this tutorial: \url{https://d2l.ai/chapter_recurrent-modern/lstm.html}.
    
    
\subsection{Deepwalk and Node2vec}
    Deepwalk and Node2vec are unsupervised methods to learn node embedding.
    Suppose we have a graph $G=(V, E)$ with $\ndata$ nodes.
    We would like to learn an embedding $\bm{z}_i \in \R^\nfeat$ for each node $i$, such that $\bm{z}^{\top}_i \bm{z}_j$ is large when $i$ and $j$ are connected.
    Let $f(i): V \rightarrow \R^\nfeat$ be the function mapping a node to its embedding.
    We represent the neighborhood (not necessarily 1-hop neighbors) of a node $i$ by $\mathcal{N}_{s}(i)$, where $s$ is the  strategy that we use to generate the neighborhood (e.g., random walk).
    Deepwalk and Node2vec solve the following maximum likelihood estimation:
        \begin{equation}
            \max_{f} \sum_{i=1}^{\ndata}{ \log p\left(\mathcal{N}_s(i) | f(i) \right)} \approx \sum_{i=1}^{\ndata}{ \sum_{j \in \mathcal{N}_s(i)}^{}{\log p(j | f(i))} },
        \end{equation}
    where the approximation utilizes the conditional independence assumption.
    The probability $p(j | f(i))$ is usually parameterized as follows:
        \begin{equation}
            p(j | f(i)) = \frac{\EXP{\bm{z}_j^\top \bm{z}_i}}{\sum_{k\in V}^{}{\EXP{\bm{z}_k^\top \bm{z}_i}}}.
        \end{equation}
    Intuitively, if nodes $j$ frequently occur in random walks started from $i$, the two nodes should be similar in some sense with each other; the softmax is used to capture the similarity.
    Computing the softmax above is very expensive due to the summation in the denominator.
    One solution is to utilize the idea of \emph{negative sampling} to do approximation:
        \begin{equation}
            \log \frac{\EXP{\bm{z}_j^\top \bm{z}_i}}{\sum_{k\in V}^{}{\EXP{\bm{z}_k^\top \bm{z}_i}}} \approx \log\left( \sigma(\bm{z}_j^\top \bm{z}_i) \right)  - \sum_{k\in \mathcal{M}}^{}{\log\left( \sigma(\bm{z}_k^\top \bm{z}_i) \right)},
        \end{equation}
    where $\sigma(\cdot)$ is the sigmoid function.
    The set $\mathcal{M}$ is a small subset of nodes sampled from the graph with probability proportional to nodes' degrees. 
    
    
    \subsection{GNN}
        Two key operations in a GNN are:
        \begin{itemize}
            \item Message passing: a node $u$ generates some messages from its embedding, i.e., $\bm{m}_u^{(l)} = \text{MSG}^{(l)}(\bm{h}_u^{(l-1)})$. The operation $\text{MSG}^{(l)}$ can be as simple as matrix multiplication, e.g., $\bm{W}^{(l)} \cdot \bm{h}^{(l-1)}_u$.
            \item Aggregation: a node $v$ generates its embedding at layer $l$ by aggregating the messages from its neighbors, i.e., $\bm{h}_v^{(l)} = \text{AGG}^{(l)}\left( \Set{\bm{m}_u^{(l)} }{u \in \mathcal{N}_v} \right)$.
        \end{itemize}
        
        \paragraph{Over-smoothing.} The embedding of a node depends on a concept called \emph{receptive field}, which is the set of nodes where messages are aggregated (e.g., the $k$-hop neighborhood).
        Intuitively, over-smoothing means that the receptive fields of each node are highly overlapped with each other, such that the nodes' embeddings are converging to the same values.
        Over-smoothing can result from stacking too many layers of GNN.
        Some solutions include:
            \begin{itemize}
                \item Limit the number of layers; a good strategy to estimate the number of layers needed is to compute the diameter of the graph.
                \item Increase the expressive power, e.g., adding some MLP layers before/after the GNN, or resorting to some attention mechanisms.
                \item Add skip connections.
            \end{itemize}
            
            
\subsection{Dropout}
    Intuitively, we zero out some neurons during training with probability $p$.
    However, we need to adjust the outputs of the remained neurons such that their expectations do not change.
    For any neuron $h$, the adjusted output is as follows:
        \begin{equation}
            h' = \begin{cases}
                    0, & \text{with prob. $p$} \\
                    \frac{h}{1-p}, & \text{with prob. $1-p$}.
                 \end{cases}
        \end{equation}
    As a result, the expected output is $\EXP{h'} = h$, which is the same as if there were no dropout.
    When testing we do not zero out any neuron and the forward pass is the same as usual.
    
\subsection{Batch Normalization (BN)}
    BN is used to center and re-scale the inputs to each layer based on statistics computed from mini-batches, which may speed-up the convergence.
    Formally, BN is the following operations:
        \begin{equation}
            \text{BN}(\bm{x}) = \bm{\gamma} \odot \frac{\bm{x} - \hat{\bm{\mu}} }{\hat{\bm{\sigma}}} + \bm{\beta},
        \end{equation}
    where $\hat{\bm{\mu}}$ and $\hat{\bm{\sigma}}$ are sample mean and standard deviation estimated from the mini-batch.
    Notice that when training the mean and standard deviation of BN are computed from mini-batches, however, when testing those statistics are computed from the entire dataset.
        
    