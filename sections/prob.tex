

\subsection{Two Types of Sampling with Replacement}
The materials are based on Chapter 1 of~\cite{degroot2012probability}.
There are two types of ``sampling with replacement":
\begin{itemize}
    \item Ordered sampling with replacement: suppose we have a box containing \ndata balls numbered from $1$ to $\ndata$.
    If we would like to select $k$ balls with replacement from the box, the number of possible ways is $\ndata^k$.
    This is because the balls are numbered so the order matters. The first selection has $\ndata$ possibilities; the second has $\ndata$ possibilities, and so on.
    
    \item Unordered sampling with replacement: suppose we go to a bakery store and would like to choose a box of $k$ cakes to go.
    The store has \ndata types of cakes.
    The number of ways to fill the box is $\binom{\ndata+k-1}{k}$.
    Notice that the answer is not $\ndata^k$, as the order of the same cake does not matter.
\end{itemize}