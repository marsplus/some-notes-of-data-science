

\subsection{Probability Integral Transform}
    This transform allows us to sample from \emph{any} continuous distribution by sampling from the uniform distribution on $[0, 1]$.
    Suppose that a continuous random variable $X$ has CDF $F_X(x)$ that is continuous and strictly increasing. 
    Define the random variable $Y=F_X(X)$, then $Y$ follows the uniform distribution on $[0, 1]$, i.e., 
        \begin{equation}
            Y \sim U[0, 1].
        \end{equation}
    This is because 
        \begin{equation}
            F_Y(y) = P(Y \le y) = P(F_X(X) \le y) = P(X \le F^{-1}_X(y)) = F_X\left(F^{-1}_X(y) \right) = y,
        \end{equation}
    which is the CDF of $U[0, 1]$.
    A direct implication is that the random variable $F^{-1}_X(Y)$ has the same distribution as $X$.
    
    \paragraph{Example 1.}
        Let $X \sim \text{Exp}(1)$. The CDF is $F_X(x) = 1 - e^{-x}$. The inverse CDF of $X$ is $F^{-1}_X(x) = -\log(1-x)$.
        To sample $X$  we  first sample $Y \sim U[0, 1]$, then the transformed random variable $-\log(1-Y)$ follows the same distribution as $X$.

\subsection{Two Types of Sampling with Replacement}
The materials are based on Chapter 1 of~\cite{degroot2012probability}.
There are two types of ``sampling with replacement":
\begin{itemize}
    \item Ordered sampling with replacement: suppose we have a box containing \ndata balls numbered from $1$ to $\ndata$.
    If we would like to select $k$ balls with replacement from the box, the number of possible ways is $\ndata^k$.
    This is because the balls are numbered so the order matters. The first selection has $\ndata$ possibilities; the second has $\ndata$ possibilities, and so on.
    
    \item Unordered sampling with replacement: suppose we go to a bakery store and would like to choose a box of $k$ cakes to go.
    The store has \ndata types of cakes.
    The number of ways to fill the box is $\binom{\ndata+k-1}{k}$.
    Notice that the answer is not $\ndata^k$, as the order of the same cake does not matter.
\end{itemize}