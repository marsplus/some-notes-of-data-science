


    
\subsection{Policy Gradient}
Let $\tau$ be a trajectory, i.e., $\tau = (s_1, a_1, \ldots, s_T, a_T)$.
The probability of generating $\tau$ is:
    \begin{equation}
        p_{\theta}(\tau) = p(s_1) \prod_{t=1}^{T}{\pi_{\theta}(a_t | s_t) p(s_{t+1} | s_t, a_t)},
    \end{equation}
where $\pi_{\theta}$ is the policy parameterized by $\theta$.
The goal of \emph{policy gradient} is to find an optimal $\theta^\ast$ such that:
    \begin{equation}
        \theta^\ast = \argmax_{\theta} \E_{\tau \sim p_\theta(\tau)}\left[ \sum_{t}^{}{r(s_t, a_t)} \right].
    \end{equation}
    
Suppose we have collected a set of trajectories $\SET{\tau_i}_{i=1}^{\ndata}$ by running the policy $\pi_{\theta}$ multiple times.
% We approximate the expected reward as follows:
%     \begin{equation}
%         \E_{\tau \sim p_\theta(\tau)}\left[ \sum_{t}^{}{r(s_t, a_t)} \right] \approx \frac{1}{\ndata}\sum_{i=1}^{\ndata}{  \sum_{t=1}^{T}{ r(s_{i,t}, a_{i,t})}      }.
%     \end{equation}
A convenient identify is below:
    \begin{equation}
        p_{\theta}(\tau)\nabla_{\theta}\log p_{\theta}(\tau) = \nabla_\theta p_\theta(\tau).
    \end{equation}
Let $J(\theta) = \E_{\tau \sim p_\theta(\tau)}\left[ \sum_{t}^{}{r(s_t, a_t)} \right]$.
The gradient of $J(\theta)$ w.r.t. $\theta$ is below:
    \begin{equation}
        \begin{aligned}
        \nabla_\theta J(\theta) = \nabla_\theta \left[ \int p_\theta(\tau) r(\tau) d\tau \right] = \int \nabla_\theta p_\theta(\tau) r(\tau) d\tau  & = \int p_{\theta}(\tau)\nabla_{\theta}\log p_{\theta}(\tau)  r(\tau) d\tau \\
        & = \E_{\tau \sim p_\theta(\tau)} \left[ \nabla_\theta \log p_\theta(\tau) r(\tau) \right].
        \end{aligned}
    \end{equation}
The loglikelihood of $\tau$ is as follows (omitting the terms that are irrelevant with $\theta$):
    \begin{equation}
        \log p_\theta(\tau) \approx \sum_{t=1}^{T}{\log \pi_{\theta}(a_t | s_t)}.
    \end{equation}
Thus, we have:
    \begin{equation}
        \nabla_\theta J(\theta) = \E_{\tau \sim p_\theta(\tau)}\left[ \left( \sum_{t=1}^{T}{\nabla_\theta \log\pi_{\theta}(a_t | s_t)}\right) \left( \sum_{t=1}^{T}{r(s_t, a_t)} \right)\right].
    \end{equation}
    
\subsubsection{On-Policy Policy Gradient}
The classic algorithm to implement the online policy gradient is called \emph{REINFORCE}:
    \begin{itemize}
        \item[1.] Sample $\SET{\tau_i}_{i=1}^{\ndata}$ by running $\pi_{\theta}$ for \ndata times.
        \item[2.] Compute $\nabla_\theta J(\theta) \approx \frac{1}{\ndata}\sum_{i=1}^{\ndata}{
        \left(\sum_{t=1}^{T}{\nabla_\theta\log \pi_\theta(a_t^i | s_t^i)}\right)
        \left(\sum_{t=1}^{T}{r(s_t^i, a_t^i)} \right)
        }$.
        \item[3.] Update $\theta \leftarrow \theta + \alpha \nabla_\theta J(\theta)$.
    \end{itemize}
    
    
    
\subsubsection{Off-Policy Policy Gradient}
Recall that \emph{Importance Sampling} is a useful technique to sample from a intractable distribution $p(x)$ by using an auxiliary distribution $q(x)$, i.e., 
    \begin{equation}
        \E_{x \sim p(x)}[f(x)] = \E_{x \sim q(x)}\left[ \frac{p(x)}{q(x)}f(x) \right].
    \end{equation}



\subsubsection{Trust Region Policy Optimization (TRPO)}



