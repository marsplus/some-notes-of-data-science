


\subsection{Checking Non-Stationarity}
A stationary sequence should exhibit the following four properties: 1) constant means; 2) constant variance; 3) constant auto-correlation structures\footnote{E.g., today's measurements are highly correlated with some past values.}; 4) no periodic component. 

There are a couple of ways to check if there is non-stationarity in the data.
    \begin{itemize}
        \item Just plot the sequence. We should check if there is any specific pattern. 
        This is usually the first step of any subsequent analysis.
        \item Divide a sequence into several chunks and compute some statistics (e.g., mean, variance, etc.) on each chunk. 
        If the statistics vary a lot, it's likely there is some non-stationarity. 
        \item Visualize a sequence as a histogram plot. The histogram should look like a normal distribution if the sequence is stationary.
        \item Use the Augmented Dickey-Fully (ADF) test. This is usually good with large datasets.
    \end{itemize}

Some transformations can be applied to convert non-stationary sequences to stationary ones:
    \begin{itemize}
        \item Periodicity or non-constant mean: remove the periodicity and make the mean constant.
        \item Heterogeneous variances: a log transformation can be applied.
        \item Auto-correlation structures: 1) determine the lag $l$; 2) minus the current value by the value at $l$ steps before, i.e., $v_t \leftarrow v_t - v_{t-l}$.
    \end{itemize}


\paragraph{Autocorrelation coefficient.}
    \begin{equation}
        r_k = \frac{\sum_{t=k+1}^{T}{(y_t - \bar{y})(y_{t-k} - \bar{y})}}{\sum_{t=1}^{T}{(y_t - \bar{y})^2}}.
    \end{equation}


\paragraph{The coefficient of determination.}
    Notice that this coefficient only measure how well the model fits the historical data. 
    \begin{equation}
        R^2 = \frac{\sum_{}^{}{(\hat{y}_t - \bar{y})^2 }}{\sum_{}^{}{(y_t - \bar{y})^2}}.
    \end{equation}
    One issue with $R^2$ is that it always increase when adding extra features to the model. The adjusted $R^2$ below avoids this issue.
    \begin{equation}
        \text{Adjusted $R^2$} = 1 - (1 - R^2)\frac{T-1}{T-k-1},
    \end{equation}
    where $T$ is the number of observations and $k$ is the number of features. 
    

